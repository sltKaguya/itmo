\documentclass[a4paper,12pt]{article}

\usepackage[utf8]{inputenc}
\usepackage[russian]{babel}
\usepackage{amsmath}
\usepackage{amssymb}

\usepackage{geometry}
\geometry{
    left=2cm,
    right=2cm,
    top=2cm,
    bottom=2cm
}

\usepackage{amsthm}
\newtheorem*{defin}{Определение}
\newtheorem*{lemma}{Лемма}

\title{ДИСКРЕТНАЯ МАТЕМАТИКА\\
    \large II-ой семестр}
\author{sltKaguya\\
    Группа M3104}
\date{8 февраля 2022 г. --- \today}

\begin{document}
\pagenumbering{gobble}
\maketitle
\newpage
\pagenumbering{arabic}

\subparagraph*{Что добавить}
\begin{enumerate}
    \item Более общее (мне пофиг, пишу как хочу) определение графа, и добавить снизу главу про неориентированные;
    \item Поправить обозначения отдельных вершин и рёбер.
\end{enumerate}

\begin{center}
    \section*{Введение в теорию графов}
\end{center}

\begin{defin}
    \textbf{Граф} $G(V, E)$ -- множество вершин $V$ и рёбер $E$, такое, что $E$ является подмножеством множества двухэлеменных подмножеств множества $V$ (для\newline
    неориентированного графа).
\end{defin}

\begin{defin}
    \textbf{Ребро} -- неупорядоченная пара $\{u, v\}$, где $u, v \in V$ и $u \not=v$ (для\newline
    неориентированного графа).
\end{defin}

Если $x$ -- ребро с концами $u$ и $v$, то $x$ \textbf{инцедентен} $u$ и $v$, $u$ и $v$ \textbf{инцедентны} $x$.

Вершины $u$ и $v$ называются смежными, если являются концами одного ребра.

Рёбра $x$ и $y$ называются смежными, если имеют общую вершину.

\begin{defin}
    \textbf{Взвешенный} граф -- граф с весами на рёбрах, то есть каждое ребро графа имеет числовое значение. Пример: расстояние, цена и так далее.
\end{defin}

\begin{defin}
    \textbf{Тривиальный} граф -- граф из одной вершины: $G(V, \varnothing), \ |V| = 1$.
\end{defin}

\begin{defin}
    \textbf{Пустой} граф или нуль граф -- граф без рёбер: $G(V, \varnothing)$.
\end{defin}

\underline{Примечание.} В некоторых источниках в пустом графе нет даже вершин: $G(\varnothing, \varnothing)$.

\subsection*{Ориентированный граф}

\begin{defin}
    \textbf{Ориентированный} граф -- множество вершин $V$ и ориентированных рёбер $E$.
\end{defin}

\begin{defin}
    \textbf{Ориентированное} ребро -- упорядоченная пара $(u, v)$, где $u, v \in V$.
\end{defin}

\begin{defin}
    \textbf{Направленный} граф -- граф без симметричных пар ориентированных рёбер, то есть такой пары рёбер $x$ и $y$, что если $x(u, v)$, то $y(v, u)$.
\end{defin}

\begin{defin}
    \textbf{Кратные} (параллельные) рёбра $x(u, v)$ и $y(u, v)$ -- рёбра, соединяющие одни и те же вершины.
\end{defin}

\begin{defin}
    \textbf{Петля} $x = (a, a)$ -- ребро, соединяющее вершину саму с собой.
\end{defin}

\underline{Примечание.} Для ориентированных графов эти понятия более естественны, так как не противоречат определению.

\begin{defin}
    \textbf{Висячая} вершина -- вершина, в которую ведёт только одно ребро. Это ребро тоже называется \textbf{висячим}.
\end{defin}

\begin{defin}
    \textbf{Изолированная} вершина -- вершина, в которую не ведёт ни одно ребро (петля).
\end{defin}

\begin{defin}
    \textbf{Простой} граф -- граф без параллельных рёбер и петель.
\end{defin}

\begin{defin}
    \textbf{Псевдограф} -- граф с параллельными рёбрами и петлями.
\end{defin}

\begin{defin}
    \textbf{Мультиграф} -- псевдограф без петель.
\end{defin}

\subsection*{Степень вершины}

\begin{defin}
    \textbf{Степень} вершины в неориентированном графе $G(V, E)$ $v_i$ -- число рёбер, инцидентных с ней.
\end{defin}

\begin{defin}
    \textbf{Степень входа} вершины $\deg^+ v_i$ -- чило рёбер, входящих в вершину.
\end{defin}

\begin{defin}
    \textbf{Степень исхода} вершины $\deg^- v_i$ -- число рёбер, исходящих из вершины.
\end{defin}

\begin{lemma}
    \textbf{(о рукопожатиях).} \begin{enumerate}
        \item Для неориентированных графов:\newline 
        Сумма степеней всех вершин графа -- чётное число, равное удвоенному числу рёбер графа: $\sum\limits_{v \in V(G)} \deg v = 2|E(G)|$\newline
        \underline{Доказательство:}\newline
        Для пустого графа сумма степеней всех вершин равна $0$.\newline
        При добавлении нового ребра у двух вершин степень увеличивается на $1$, то есть суммарно на 2.\newline
        При добавлении $n$ новых рёбер сумма степеней увеличится на $2n$, что и требовалось доказать.\newline
        \underline{Следствие.} Произвольный граф содержит чётное число вершин нечётной степени.
        \underline{Доказательство:}\newline
        $V_0(G)$ - множество вершин чётной степени, $V_1(G)$ -- множество вершин нечётной степени.\newline
        $\sum\limits_{v \in V_0(G)} \deg v + \sum\limits_{u \in V_1(G)} \deg u = 2|E(G)|$. Первое слагаемое -- чётное число, потому что сумма чётных чисел всегда будет чётной. Результат тоже будет чётным. Слагаемое и сумма чётны, значит, $\sum\limits_{u \in V_1(G)} \deg u$ -- тоже чётное число. Значит, количество нечётных $\deg u$ должно быть чётным, то есть $|V_1(G)|$ -- чётно, что и требовалось доказать.
        \item Для ориентированных графов:\newline
        Cумма входящих и исходящих степеней всех вершин графа -- чётное число, равное удвоенному числу рёбер графа: $\sum\limits_{v \in V} \deg^+ v + \sum\limits_{v \in V} \deg^- v = 2|E|$\newline
        \underline{Доказательство.}\newline
        Для пустого графа сумма входящих и исходящих степеней всех вершин равна 0.\newline
        При добавлении нового ребра у одной вершины степень исхода увеличится на $1$, а у другой -- степень входа на $1$, то есть суммарно на 2.\newline
        При добавлении $n$ рёбер сумма входящих и исхидящих степеней увеличится на $2n$, что и требовалось доказать.
    \end{enumerate}
\end{lemma}

\subsection*{Способы представления графа}

\begin{enumerate}
    \item Диаграмма -- схематичный рисунок графа, где вершины -- точки, рёбра -- соединяющие их отрезки или дуги.
    \item Список смежности -- удобен для разреженных графов, у которых мало рёбер.
    \begin{enumerate}
        \item Неориентированный граф будет представлять из себя список (массив), каждый элемент которого -- вершина графа, содержащая ссылки на каждую смежную с ней вершину. Сумма длин всех списков равна удвоенному числу рёбер. Объём используемой памяти -- $\Theta(|E| + |V|)$.
        \item Ориентированный граф будет представлять собой примерно то же, что и неориентированный, но только в нём не будут дублироваться ссылки на смежные вершины, а будут присутствовать в единственном варианте. Сумма длин всех списков равна числу рёбер. Объем используемой памяти -- $\Theta(|E| + |V|)$.
    \end{enumerate}
    \item Матрица смежности -- удобна для плотных графов, у которых много рёбер. Представляет из себя матрицу $A_{|V| \times |V|}$, в каждой ячейке которой записано количество рёбер, соединяющих вершины.
    \begin{enumerate}
        \item В неориентированном графе петля учитывается дважды. Матрица будет симметрична. Значит, объём иcпользуемой памяти $\Theta(|V|^2)$ можно уменьшить до $\Theta\left(\frac{|V|^2}{2}\right)$, оставив только главную диагональ и всё, что выше неё. Сумма значений в столбце или строке будет степенью вершины.
        \item В ориентированном графе петля учитывается единожды. Объём используемой памяти -- $\Theta(|V|^2)$. Сумма значений в строке будет степенью исхода вершины, а в столбце -- степенью входа.
    \end{enumerate}
    Свойства:
    \begin{enumerate}
        \item В простом графе -- бинарна;
        \item В простом графе -- главная диагональ состоит из нулей;
        \item В неориентированном графе -- симметрична относительно главной диагонали;
        \item В ориентированном графе -- сумма элементов в строке равна степени входа вершины, в столбце -- исхода вершины.
    \end{enumerate}
    Во взвешенном графе вместо 1 хранится вес ребра, вместо 1 -- nil.
    \item Матрица инцедентности -- удобна для графов с кратными рёбрами и петлями. Представляет из себя матрицу $I_{|V| \times |E|}$
    \begin{enumerate}
        \item 
    \end{enumerate}
\end{enumerate}
\end{document}