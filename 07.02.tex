\documentclass[a4paper,12pt]{article}

\usepackage[utf8]{inputenc}
\usepackage[russian]{babel}
\usepackage{amsmath}
\usepackage{amssymb}

\usepackage{geometry}
\geometry{
    left=2cm,
    right=2cm,
    top=2cm,
    bottom=2cm
}

\title{ПРОГРАММИРОВАНИЕ\\
    \large II-ой семестр}
\author{sltKaguya\\
    Группа M3104}
\date{7 февраля 2022 г. --- \today}

\begin{document}
\maketitle
\newpage
\pagenumbering{arabic}
\begin{center}
    \section*{ООП, классы}
\end{center}
\subsection*{Классы}

\qquadКласс -- это тип данных, который состоит из свойств и методов. Также можно сказать, что это шаблон для создания объектов. Свойства класса -- любые данные, которые характеризуют объект класса. Методы класса -- привязанные к этому классу функции, выполняющие действия над данными (свойствами) класса.

Класс в c++ объявляется следующим образом:
\vspace{6pt}\newline
\textbf{class} ClassName \{\newline
some\_variable\_1;\newline
some\_variable\_2;\newline
some\_function();\newline
\}
\end{document}