\documentclass[a4paper,12pt]{article}

\usepackage[utf8]{inputenc}
\usepackage[russian]{babel}
\usepackage{amsmath}
\usepackage{amssymb}

\usepackage{geometry}
\geometry{
    left=2cm,
    right=2cm,
    top=2cm,
    bottom=2cm
}

\title{ПРОГРАММИРОВАНИЕ\\
    \large II-ой семестр}
\author{sltKaguya\\
    Группа M3104}
\date{7 февраля 2022 г. --- \today}

\begin{document}
\maketitle
\newpage
\pagenumbering{arabic}

\subparagraph*{Что добавить}
\begin{enumerate}
    \item std;
    \item Типы данных;
    \item Ввод-вывод;
    \item Функции и методы.
\end{enumerate}

\begin{center}
    \section*{Начало работы с c++}
\end{center}

\subsection*{Namespace std}
\qquad\textbf{std} -- стандартное (\textbf{st}andar\textbf{d}) пространство

\textbf{using namespace} и оператор \textbf{::}

\subsection*{Типы данных}

\subsection*{Ввод-вывод}

\subsection*{Основные функции и методы}

\begin{itemize}
    \item getline(std::cin, foo) -- объектом cin записывает текст из командной строки в строку foo. Позволяет получить строку с пробелами, тогда как просто cin прекратит считывать после пробела
\end{itemize}

\begin{center}
    \section*{ООП, классы}
\end{center}

\subsection*{Классы}

\qquadКласс -- это тип данных, который состоит из свойств и методов. Также можно сказать, что это шаблон для создания объектов. Свойства класса -- любые данные, которые характеризуют объект класса. Методы класса -- привязанные к этому классу функции, выполняющие действия над данными (свойствами) класса.

Класс в c++ объявляется следующим образом:
\vspace{6pt}\newline
\textbf{class} ClassName \{\newline
some\_variable\_1;\newline
some\_variable\_2;\newline
ClassName some\_method()\{\};\newline
\}

\subsection*{Конструкторы}

\qquadКонструктор -- это метод, который выполняется при объявлении или инициализации класса. В отличие от метода, он прописывается как:
\vspace{6pt}\newline
\textbf{class} ClassName \{\newline
ClassName(variables) \{\newline
code\newline
\}\newline
\}
\end{document}